\newpage
\begin{center}
  \textbf{\large 1. Маркетмейкер, использующий константное произведение }
\end{center}
\refstepcounter{chapter}
\addcontentsline{toc}{chapter}{1. МАРКЕТМЕЙКЕР, ИСПОЛЬЗУЮЩИЙ КОНСТАНТНОЕ ПРОИЗВЕДЕНИЕ}

\section{Использование маркетмейкеров в децентрализованных биржах}

Изначально децентрализованные биржи представляют собой метод обмена токенами между участниками блокчейн-протокола без необходимости доверять какому-либо централизованному источнику информации о ценах, объемах торгов и прочего. На текущий момент (конец 2025 года) порядка 200 миллиардов долларов участвовало в различных протоколах децентрализованного обмена. 

В общем виде, ключевым объектом децентрализованных бирж является маркетмейкер, основанный на константной функции (CFMM, constant function market makers). Успех Uniswap в свое время способствовал развитию популярности автоматизированных маркетмейкеров (AMM, automated market maker), основанных на функции константного произведения (constant product market maker). Маркет мейкер, основанный на константной функции, можно задать с помощью функции торговли $\varphi$:

\begin{equation}
\varphi: \mathbb{R}^n_{+} \times \mathbb{R}^n_{+} \times \mathbb{R}^n_{+} \rightarrow \mathbb{R}
  \label{eq1}
\end{equation}

Функция трех аргументов $\varphi$ формально задает процесс изменения состояния т.н. пула ликвидности (Liquidity Pool). Первый аргумент представляет собой вектор резервов $\mathbf{R} \subset \mathbb{R}^n_{+}$. Второй аргумент представляет собой вектор входов $\mathbf{\delta} \subset \mathbb{R}^n_{+}$, т.е. то, сколько токенов предоставил пользователь в пул ликвидности в процессе торговли. Третий аргумент представляет собой вектор выходов $\mathbf{\lambda} \subset \mathbb{R}^n_{+}$, т.е. то, сколько токенов получит пользователь от пула ликвидности в процессе торговли.
Возвращает функция количество ликвидности после совершения сделки.

Приведем конкретный пример. Пусть задан пул ликвидности Uniswap V3, в нем находятся токены WETH(обернутый Ethereum, ERC-20 адаптация нативного токена сети Ethereum Mainnet) и USDT (Tether USD, стейблкоин, чья стоимость примерно равна одному доллару США). В случае Uniswap V3, количество токенов в пуле $n = 2$

Эти токены находятся в пуле в определенном количестве, и это количество задает вектор резервов, например: $\mathbf{R} = \{1, 2000\}$. В пуле находятся 2000 USDT и 1 WETH.
Векторы $\delta$ и $\lambda$ задают т.н. сделку, которая состоит из вектора входов и вектора выходов. Например, если пользователь хочет купить WETH, предоставив 300 USDT, то эта сделка будет выглядеть так: предоставление 300 USDT $\delta = \{0, 300\}$ и получение взамен 0.1304 WETH $\lambda = \{0, 0.1304\}$

Почему именно 0.1304 WETH? Дело в том, что на функцию $\varphi$ накладывается важное ограничение, которое и называется постоянным произведения (constant product). Сделка $(\delta, \lambda)$ считается валидной, если

\begin{equation}
\varphi(R, \delta, \lambda) = \varphi(R, \mathbf{0}, \mathbf{0})
  \label{eq2}
\end{equation}

Иными словами, сделка считается валидной, если общее количество ликвидности не меняется после совершения сделки. В нашем примере со сделкой покупки WETH за 300 USDT, количество WETH, которое получит пользователь, задается именно этим правилом.

\section{Задание функции торговли}
Чтобы определять, сколько пользователь получит токенов в результате совершения сделки, нужно задать функцию торговли.

Функция торговли для Uniswap V3 выглядит следующим образом

\begin{equation}
\varphi(R, \delta, \lambda) = (R_1 + \delta_1 - \lambda_1) \cdot (R_2 + \delta_2 - \lambda_2)
  \label{eq3}
\end{equation}

где 
\begin{enumerate}
\item $R_1$ и $R_2$ представляют собой компоненты вектора резервов $\mathbf{R}$, количество токенов WETH и USDT в пуле соответственно;
\item $\delta_1$ и $\delta_2$ представляют собой количество WETH и USDT, которые пользователь предоставляет в пул в процессе сделки;
\item $\lambda_1$ и $\lambda_2$ представляют собой количество WETH и USDT, которые пользователь получит в результате сделки.
\end{enumerate}

В случае Uniswap V3 можно считать, что $\delta_1$ и $\delta_2$, равно как и $\lambda_1$ и $\lambda_2$ не могут быть одновременно ненулевыми.
Для целого ряда сделок вида "пользователь предоставляет USDT взамен на WETH" мы можем упростить функцию торговли:

\begin{equation}
\varphi(R, \delta, \lambda) = (R_1 - \lambda_1) \cdot (R_2 + \delta_2)
  \label{eq4}
\end{equation}

В рамках таких сделок количество токена с индексом $1$ (WETH) уменьшается на $\lambda_1$, а количество токена с индексом $2$ (USDT) увеличивается на $\delta_2$.

\section{Пример сделки}
Стоит вернуться к примеру сделки, где пользователь покупает WETH за 300 USDT. 
До сделки количество ликвидности в пуле составляет:

\begin{equation}
\varphi(R, \mathbf{0}, \mathbf{0}) = (1 - 0) \cdot (2000 + 0) = 2000
  \label{eq5}
\end{equation}

После сделки, согласно $\eqref{eq2}$, количество ликвидности должно остаться также $2000$.
Подставим в $\eqref{eq4}$ количество предоставляемых пользователем токенов USDT: $\delta_2 = 300$ и получим количество WETH $\lambda_1$, которые пользователь получит взамен:

\begin{equation}
\begin{gathered}
\varphi(R, \delta, \lambda) = (R_1 - \lambda_1) \cdot (R_2 + 300) = 2000
\\
(1 - \lambda_1) \cdot (2000 + 300) = 2000
\\
\lambda_1 = 1 - \frac{2000}{2300}
\\
\lambda_1 = 0.1304
\end{gathered}
\end{equation}
\section{Торговое множество и достижимость сделки}

Следующее развитие мысли -- введение понятия торгового множества $T$, которое представляет собой множество сделок, доступных в конкретный момент времени (точнее, для конкретного набора резервов токенов $\mathbf{R}$)

\begin{equation}
T(\mathbf{R}) = \{(\delta, \lambda) | \varphi(\mathbf{R}, \delta', \lambda') = 0\}
  \label{eq6}
\end{equation}

где сделка $(\delta', \lambda')$ -- произвольная сделка, которая "выгоднее" для пользователя, чем сделка $(\delta, \lambda)$, которая попадает в множество доступных сделок $T(R)$.
Под "выгодностью" сделки понимается, что $\delta' \leq \delta$, а $\lambda' \geq \lambda$, т.е. в пересчете на одну единицу предоставляемых пользователем токенов $\delta$ пользователь получит больше токенов $\lambda$.

Таким образом, торговое множество состоит из таких сделок, выгоднее которых нет для пользователя в момент времени, когда в пуле ликвидности находятся активы $\mathbf{R}$.
Идея заключается в том, что пользователя можно представить как рационального агента, которые из двух и более сделок будет выбирать ту, которая для него будет наиболее выгодной.

\section{Цели и задачи магистерской работы}


\textbf{Цель работы} -- установить связь дальнодействия притяжения потенциала взаимодействия и спектров возбуждений с транспортными свойствами жидкостей, а также влияние на скорость нуклеации.

\textbf{Задачи работы:}
\begin{enumerate}
\item Расчет фазовых диаграмм для 2D и 3D систем частиц, взаимодействующих посредством обобщенного потенциала Леннарда-Джонса с различными степенями притяжения.
\item Адаптация метода кластеризации данных DBSCAN для изучения молекулярных систем и его сравнение с другими методами.
\item Расчет и анализ транспортных свойств и коллективных возбуждений на жидкостных бинодалях.
\item Применение нового метода распознавания фаз для изучения скорости нуклеации в переохлажденных системах Леннарда-Джонса с различным дальнодействием притяжения.
\end{enumerate}
